\documentclass[a4paper,12pt]{article}

\usepackage{worksheet}

\usepackage{cellspace}

\usepackage[x11names]{xcolor}

\usepackage{tikz}

\begin{document}

% \large
\section*{Sheet Title}

\subsection*{Question Bank}

\textsc{Note}: Any questions where you get a decimal or fraction 
as an answer can be rounded to 2 decimal places or left as 
an exact value.

\begin{enumerate}
\item For each of the following equations, fill out the corresponding
    table of values, plot the points on a Cartesian plane, and 
    draw a graph of the curve. Does the graph converge on a value 
    as $x\to\infty$ or $x\to -\infty$?
    \begin{enumerate}
    \item $y = 2^x$

        \begin{tabular}{|Sc|Sc|Sc|Sc|Sc|Sc|Sc|Sc|Sc|Sc|Sc|Sc|Sc|Sc|Sc|Sc|Sc|Sc|Sc|Sc|Sc|Sc|Sc|Sc|Sc|Sc|Sc|Sc|Sc|Sc|Sc|Sc|Sc|Sc|Sc|Sc|Sc|Sc|Sc|Sc|Sc|Sc|Sc|Sc|Sc|Sc|Sc|Sc|Sc|Sc|Sc|Sc|Sc|Sc|Sc|Sc|Sc|Sc|Sc|Sc|Sc|Sc|Sc|Sc|Sc|Sc|Sc|Sc|Sc|Sc|Sc|Sc|Sc|Sc|Sc|Sc|Sc|Sc|Sc|}
        \hline
        $x$ & $-3$ & $-2$ & $-1$ & 0 & 1 & 2 & 3 \\
        \hline
        $y$ & & & & & & & \\[3mm]
        \hline
        \end{tabular}
    \item $y = 2^{-x}$

        \begin{tabular}{|Sc|Sc|Sc|Sc|Sc|Sc|Sc|Sc|Sc|Sc|Sc|Sc|Sc|Sc|Sc|Sc|Sc|Sc|Sc|Sc|Sc|Sc|Sc|Sc|Sc|Sc|Sc|Sc|Sc|Sc|Sc|Sc|Sc|Sc|Sc|Sc|Sc|Sc|Sc|Sc|Sc|Sc|Sc|Sc|Sc|Sc|Sc|Sc|Sc|Sc|Sc|Sc|Sc|Sc|Sc|Sc|Sc|Sc|Sc|Sc|Sc|Sc|Sc|Sc|Sc|Sc|Sc|Sc|Sc|Sc|Sc|Sc|Sc|Sc|Sc|Sc|Sc|Sc|Sc|}
        \hline
        $x$ & $-3$ & $-2$ & $-1$ & 0 & 1 & 2 & 3 \\
        \hline
        $y$ & & & & & & & \\[3mm]
        \hline
        \end{tabular}
    \item $y = 2^{x - 1}$

        \begin{tabular}{|Sc|Sc|Sc|Sc|Sc|Sc|Sc|Sc|Sc|Sc|Sc|Sc|Sc|Sc|Sc|Sc|Sc|Sc|Sc|Sc|Sc|Sc|Sc|Sc|Sc|Sc|Sc|Sc|Sc|Sc|Sc|Sc|Sc|Sc|Sc|Sc|Sc|Sc|Sc|Sc|Sc|Sc|Sc|Sc|Sc|Sc|Sc|Sc|Sc|Sc|Sc|Sc|Sc|Sc|Sc|Sc|Sc|Sc|Sc|Sc|Sc|Sc|Sc|Sc|Sc|Sc|Sc|Sc|Sc|Sc|Sc|Sc|Sc|Sc|Sc|Sc|Sc|Sc|Sc|}
        \hline
        $x$ & $-3$ & $-2$ & $-1$ & 0 & 1 & 2 & 3 \\
        \hline
        $y$ & & & & & & & \\[3mm]
        \hline
        \end{tabular}
    \item $y = 2^{\frac{x}{2}}$

        \begin{tabular}{|Sc|Sc|Sc|Sc|Sc|Sc|Sc|Sc|Sc|Sc|Sc|Sc|Sc|Sc|Sc|Sc|Sc|Sc|Sc|Sc|Sc|Sc|Sc|Sc|Sc|Sc|Sc|Sc|Sc|Sc|Sc|Sc|Sc|Sc|Sc|Sc|Sc|Sc|Sc|Sc|Sc|Sc|Sc|Sc|Sc|Sc|Sc|Sc|Sc|Sc|Sc|Sc|Sc|Sc|Sc|Sc|Sc|Sc|Sc|Sc|Sc|Sc|Sc|Sc|Sc|Sc|Sc|Sc|Sc|Sc|Sc|Sc|Sc|Sc|Sc|Sc|Sc|Sc|Sc|}
        \hline
        $x$ & $-6$ & $-4$ & $-2$ & 0 & 2 & 4 & 6 \\
        \hline
        $y$ & & & & & & & \\[3mm]
        \hline
        \end{tabular}
    \end{enumerate}

\item For each of the following equations create a table of values
    and use them to graph the curves on a Cartesian plane.
    \begin{multicols}{2}
    \begin{enumerate}
    \item $y = \left(\frac{1}{2}\right)^x$
    \item $y = -3^x$
    \item $y = 2^{x+4}$
    \item $y = 3^x - 3$
    \end{enumerate}
    \end{multicols}
\item The graphs from question 1. b) and 2. a) should look the same.
    Based on their equations, $y = 2^{-x}$ and 
    $y = \left(\frac{1}{2}\right)^x$ why do you think this could be
    the case?
\item For each of the following equations, plot a graph using 
    a method you feel comfortable with. On the same plane plot the 
    line of the horizontal asymptote (the value the curve converges
    upon) and state its equation.
    \begin{multicols}{2}
    \begin{enumerate}
    \item $y = 2^x - 2$
    \item $y = -2^x + 2$
    \item $y = 3(3^{x - 2} - 1)$
    \item $y = \frac{2^{-x+2} + 6}{3}$
    \end{enumerate}
    \end{multicols}
\end{enumerate}

\newpage
\subsection*{Answers}

\begin{enumerate}
\item 
    \begin{enumerate}
        \item 
        \begin{tabular}{|Sc|Sc|Sc|Sc|Sc|Sc|Sc|Sc|Sc|Sc|Sc|Sc|Sc|Sc|Sc|Sc|Sc|Sc|Sc|Sc|Sc|Sc|Sc|Sc|Sc|Sc|Sc|Sc|Sc|Sc|Sc|Sc|Sc|Sc|Sc|Sc|Sc|Sc|Sc|Sc|Sc|Sc|Sc|Sc|Sc|Sc|Sc|Sc|Sc|Sc|Sc|Sc|Sc|Sc|Sc|Sc|Sc|Sc|Sc|Sc|Sc|Sc|Sc|Sc|Sc|Sc|Sc|Sc|Sc|Sc|Sc|Sc|Sc|Sc|Sc|Sc|Sc|Sc|Sc|Sc|}
        \hline
        $x$ & $-3$ & $-2$ & $-1$ & 0 & 1 & 2 & 3 \\
        \hline
        $y$ & $\frac{1}{8}$ & $\frac{1}{4}$ & $\frac{1}{2}$ & 1 & 2 & 4 & 8 \\
        \hline
        \end{tabular}

        As $x\to -\infty$ the curve converges toward 0.

        \item 
        \begin{tabular}{|Sc|Sc|Sc|Sc|Sc|Sc|Sc|Sc|Sc|Sc|Sc|Sc|Sc|Sc|Sc|Sc|Sc|Sc|Sc|Sc|Sc|Sc|Sc|Sc|Sc|Sc|Sc|Sc|Sc|Sc|Sc|Sc|Sc|Sc|Sc|Sc|Sc|Sc|Sc|Sc|Sc|Sc|Sc|Sc|Sc|Sc|Sc|Sc|Sc|Sc|Sc|Sc|Sc|Sc|Sc|Sc|Sc|Sc|Sc|Sc|Sc|Sc|Sc|Sc|Sc|Sc|Sc|Sc|Sc|Sc|Sc|Sc|Sc|Sc|Sc|Sc|Sc|Sc|Sc|Sc|}
        \hline
        $x$ & $-3$ & $-2$ & $-1$ & 0 & 1 & 2 & 3 \\
        \hline
        $y$ & 8 & 4 & 2 & 1 & $\frac{1}{2}$ & $\frac{1}{4}$ & $\frac{1}{8}$ \\
        \hline
        \end{tabular}

        As $x\to \infty$ the curve converges toward 0.

        \item 
        \begin{tabular}{|Sc|Sc|Sc|Sc|Sc|Sc|Sc|Sc|Sc|Sc|Sc|Sc|Sc|Sc|Sc|Sc|Sc|Sc|Sc|Sc|Sc|Sc|Sc|Sc|Sc|Sc|Sc|Sc|Sc|Sc|Sc|Sc|Sc|Sc|Sc|Sc|Sc|Sc|Sc|Sc|Sc|Sc|Sc|Sc|Sc|Sc|Sc|Sc|Sc|Sc|Sc|Sc|Sc|Sc|Sc|Sc|Sc|Sc|Sc|Sc|Sc|Sc|Sc|Sc|Sc|Sc|Sc|Sc|Sc|Sc|Sc|Sc|Sc|Sc|Sc|Sc|Sc|Sc|Sc|Sc|}
        \hline
        $x$ & $-3$ & $-2$ & $-1$ & 0 & 1 & 2 & 3 \\
        \hline
        $y$ & $\frac{1}{16}$ & $\frac{1}{8}$ & $\frac{1}{4}$ & $\frac{1}{2}$ & 1 & 2 & 4 \\
        \hline
        \end{tabular}

        As $x\to -\infty$ the curve converges toward 0.

        \item 
        \begin{tabular}{|Sc|Sc|Sc|Sc|Sc|Sc|Sc|Sc|Sc|Sc|Sc|Sc|Sc|Sc|Sc|Sc|Sc|Sc|Sc|Sc|Sc|Sc|Sc|Sc|Sc|Sc|Sc|Sc|Sc|Sc|Sc|Sc|Sc|Sc|Sc|Sc|Sc|Sc|Sc|Sc|Sc|Sc|Sc|Sc|Sc|Sc|Sc|Sc|Sc|Sc|Sc|Sc|Sc|Sc|Sc|Sc|Sc|Sc|Sc|Sc|Sc|Sc|Sc|Sc|Sc|Sc|Sc|Sc|Sc|Sc|Sc|Sc|Sc|Sc|Sc|Sc|Sc|Sc|Sc|Sc|}
        \hline
        $x$ & $-6$ & $-4$ & $-2$ & 0 & 2 & 4 & 6 \\
        \hline
        $y$ & $\frac{1}{8}$ & $\frac{1}{4}$ & $\frac{1}{2}$ & 1 & 2 & 4 & 8 \\
        \hline
        \end{tabular}

        As $x\to -\infty$ the curve converges toward 0.

    \end{enumerate}

\item
    \begin{enumerate}
    \item $y = \left(\frac{1}{2}\right)^x$

    \begin{tikzpicture}[scale=0.5]
        \draw[very thin,gray!25] (-4,-4) grid (4,4);
        \draw[thick,<->] (0,-4.25) -- (0,4.25);
        \draw[thick,<->] (-4.25,0) -- (4.25,0);

        \draw[thick] (-0.1,1) -- (0.1,1) node[right] {1};

        \draw[thick,domain=-2:4,<->,Green2] plot (\x,{exp(\x * ln(1/2))});
    \end{tikzpicture}

    \item $y = -3^x$

    \begin{tikzpicture}[scale=0.5]
        \draw[very thin,gray!25] (-4,-4) grid (4,4);
        \draw[thick,<->] (0,-4.25) -- (0,4.25);
        \draw[thick,<->] (-4.25,0) -- (4.25,0);

        \draw[thick] (-0.1,-1) -- (0.1,-1) node[right] {$-1$};

        \draw[thick,domain=-4:1.26,<->,Green2] plot (\x,{-exp(\x * ln(3))});
    \end{tikzpicture}

    \item $y = 2^{x+4}$

    \begin{tikzpicture}[scale=0.5]
        \draw[very thin,gray!25] (-7,-4) grid (1,4);
        \draw[thick,<->] (0,-4.25) -- (0,4.25);
        \draw[thick,<->] (-7.25,0) -- (1.25,0);

        \draw[thick,domain=-7:-2,<->,Green2] plot (\x,{exp((\x + 4) * ln(2))});
    \end{tikzpicture}

    \item $y = 3^x - 3$

    \begin{tikzpicture}[scale=0.5]
        \draw[very thin,gray!25] (-4,-4) grid (4,4);
        \draw[thick,<->] (0,-4.25) -- (0,4.25);
        \draw[thick,<->] (-4.25,0) -- (4.25,0);

        \draw[thick] (1,-0.1) -- (1,0.1) node[above] {1};
        \draw[thick] (-0.1,-2) -- (0.1,-2) node[right] {$-2$};

        \draw[thick,domain=-4:1.77,<->,Green2] plot (\x,{exp(\x * ln(3))-3});
    \end{tikzpicture}
    \end{enumerate}

\item You can use index laws to show that
    \begin{align*}
    y &= \left(\frac{1}{2}\right)^x, \\
      &= \frac{1^x}{2^x}, \\
      &= \frac{1}{2^x}, \\
      &= 2^{-x}.
    \end{align*}

\item
    \begin{enumerate}
    \item $y = 2^x - 2$ has an asymptote at $y = -2$

    \begin{tikzpicture}[scale=0.5]
        \draw[very thin,gray!25] (-4,-4) grid (4,4);
        \draw[thick,<->] (0,-4.25) -- (0,4.25);
        \draw[thick,<->] (-4.25,0) -- (4.25,0);

        \draw[thick,domain=-4:2.58,<->,Green2] plot (\x,{exp(\x * ln(2))-2});

        \draw[thick,<->,Firebrick2] (-4,-2) -- (4,-2) node[right] {$y = -2$};
    \end{tikzpicture}

    \item $y = -2^x + 2$
    \item $y = 3(3^{x - 2} - 1)$
    \item $y = \frac{2^{-x+2} + 6}{3}$
    \end{enumerate}

\end{enumerate}

\end{document}
