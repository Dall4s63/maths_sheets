\documentclass[a4paper,12pt]{article}

\usepackage{worksheet}

\usepackage[x11names]{xcolor}

\begin{document}
\section*{Tips for Solving Equations}
Solving an equation is a series of steps where you can apply an
operation to both sides of an equality until you have a 
equation that equates a variable with a value. It is the process
of taking a equation such as
$$4(x + 5) = 48,$$
and from this equation, determining that
$$x = 7.$$

\subsection*{Adding to Both Sides}

\subsubsection*{Example One}

The first tool in our toolbox is adding a number to both sides.
Consider the equation
$$x + 10 = 15,$$
we can add {\color{Green3}5} to both sides,
$$x + 10 \color{Green3} + 5\color{black} = 15 \color{Green3} + 5 \color{black},$$
since we can simplify $x + 10 \color{Green3} + 5$ into $x + 15$ and 
$15 \color{Green3} + 5$ into 20, the equation becomes
$$x + 15 = 20.$$
While this example isn't as interesting there are others where
this tool is useful for simplifying an equation. 

\subsubsection*{Example Two}

Consider the equation
$$x - 15 = 7,$$
we might try to add {\color{Firebrick3}15} to both sides and see
what happens,
$$x - 15 \color{Firebrick3} + 15 \color{black} = 7 \color{Firebrick3} + 15\color{black},$$
in this case we can simplify the $x - 15 \color{Firebrick3}+ 15$
into just $x$. This gives us the new equation
$$x = 22,$$
and we have solved it!

\subsubsection*{Example Three}

This can be used in more complex equations when there is a 
``lonely subtraction'' so to speak. For example something a bit
crazier such as
$$5\left(\frac{x}{2}-12\right)^2 - 22 = 13,$$
can be simplified by adding $\color{Purple3}22$ to both sides.
\begin{align*}
5\left(\frac{x}{2}-12\right)^2 - 22 \color{Purple3} + 22\color{black} &= 13\color{Purple3} + 22\color{black}, \\
5\left(\frac{x}{2}-12\right)^2 &= 35.
\end{align*}

\subsubsection*{Example Four}

There are times when adding to both sides is not as useful. 
Consider the equation
$$4(x - 8) = 12,$$
we might try to add {\color{Turquoise3}8} to both sides,
$$4(x - 8) \color{Turquoise3}+ 8\color{black} = 12 \color{Turquoise3}+ 8\color{black},$$
but unfortunately the brackets protect the $-8$ and 
$\color{Turquoise3}+8$ from being simplified. If you expand the
LHS into,
$$4x - 32 = 12,$$
you might notice that we could instead expand then do the addition.

\end{document}
