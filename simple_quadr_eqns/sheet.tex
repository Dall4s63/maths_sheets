\documentclass[a4paper,12pt]{article}

\usepackage[T1]{fontenc}
\usepackage{geometry}
\geometry{
    total={150mm,237mm},
    left=30mm,
    top=30mm,
}
\usepackage{amsmath}
\usepackage{multicol}

\setlength{\parindent}{0pt}

\begin{document}
\large
\section*{Simple Quadratic Equations}

\subsection*{Example Problems}

\textsc{Example 1}: Solve the following equation for $x$,
$$x^2 = 49.$$

\textsc{Solution}: To find the answer to the equation we need to
find the square root of both sides of the equation,
$$\sqrt{x^2} = \sqrt{49},$$
the LHS simplifies and for the RHS we can evaluate the square root,
$$x = \pm 7.$$
Note here that the $\pm$ is important since $7^2 = 49$ and 
$(-7)^2 = 49$ so we use it to indicate that $x = -7 \textrm{ or } 7$.

\vspace{3mm}
\hrule
\vspace{3mm}

\textsc{Example 2}: Solve the following equation for $x$,
$$3x^2 - 19 = 56.$$

\textsc{Solution}: In order to simplify the LHS, we first will 
remove the $-19$ since it is the last operation performed on $x$.
We can do this by adding 19 to both sides
\begin{align*}
3x^2 - 19 + 19 &= 56 + 19, \\
3x^2 &= 75.
\end{align*}
Next we will divide both sides by 3,
\begin{align*}
3x^2 \div 3 &= 75 \div 3, \\
x^2 &= 25,
\end{align*}
and finally square root both sides,
\begin{align*}
\sqrt{x^2} &= \sqrt{25}, \\
x &= \pm 5.
\end{align*}

\vspace{3mm}
\hrule
\vspace{3mm}

\textsc{Example 3}: Solve the following equation for $x$,
$$\frac{x^2 - 12}{4} = 13.$$

\textsc{Solution}: To simplify the LHS first we need to notice that
the numerator of the fraction has implicit brackets,
$$\frac{(x^2 - 12)}{4} = 13,$$
which means that we first need to multiply both sides by 4,
\begin{align*}
\frac{x^2 - 12}{4} \times 4 &= 13 \times 4, \\
x^2 - 12 &= 52,
\end{align*}
then we add 12 and square root
\begin{align*}
x^2 - 12 + 12 &= 52 + 12, \\
x^2 &= 64, \\
\sqrt{x^2} &= \sqrt{64}, \\
x &= \pm 8.
\end{align*}

\newpage
\subsection*{Question Bank}

\begin{enumerate}
\item Solve the following equations.
    \begin{multicols}{2}
    \begin{enumerate}
    \item $x^2 = 9$
    \item $x^2 = 1$
    \item $x^2 = 441$
    \item $x^2 = 169$
    \item $x^2 = 0$
    \item $x^2 = 200$
    \end{enumerate}
    \end{multicols}
\item Solve the following equations.
    \begin{multicols}{2}
    \begin{enumerate}
    \item $5x^2 = 45$
    \item $13x^2 = 637$
    \item $-3x^2 = -27$
    \item $x^2 + 14 = 95$
    \item $x^2 - 17 = 47$
    \item $x^2 - 256 = -87$
    \item $x^2 - 36 = 0$
    \item $\displaystyle\frac{x^2}{3} = 27$
    \item $\displaystyle\frac{x^2}{4} = 25$
    \item $\displaystyle\frac{x^2}{28} = \frac{9}{7}$
    \item $\displaystyle\frac{x^2}{52} = 3.25$
    \item $\displaystyle\frac{-x^2}{4} = -16$
    \end{enumerate}
    \end{multicols}
\item 
\end{enumerate}

\newpage
\subsection*{Answers}

\begin{enumerate}
\item 
    \begin{multicols}{2}
    \begin{enumerate}
    \item $x = \pm 3$
    \item $x = \pm 1$
    \item $x = \pm 21$
    \item $x = \pm 13$
    \item $x = 0$
    \item $x = \pm 14.14$ ($x = \pm 10 \sqrt{2}$ as an exact value)
    \end{enumerate}
    \end{multicols}
\item
    \begin{multicols}{2}
    \begin{enumerate}
    \item $x = \pm 3$
    \item $x = \pm 7$
    \item $x = \pm 3$
    \item $x = \pm 9$
    \item $x = \pm 8$
    \item $x = \pm 13$
    \item $x = \pm 6$
    \item $x = \pm 9$
    \item $x = \pm 10$
    \item $x = \pm 6$
    \item $x = \pm 13$
    \item $x = \pm 8$
    \end{enumerate}
    \end{multicols}
\end{enumerate}

\end{document}
