\documentclass[a4paper,12pt]{article}

\usepackage[T1]{fontenc}
\usepackage{geometry}
\geometry{
    total={150mm,237mm},
    left=30mm,
    top=30mm,
}
\usepackage{amsmath}
\usepackage{multicol}

\setlength{\parindent}{0pt}

\renewcommand{\labelenumii}{\alph{enumii})}
\everymath{\displaystyle}

\begin{document}
\large
\section*{Simple Quadratic Equations Extension}

\subsection*{Example Problems}

\textsc{Example 1}: Solve the following quadratic equation,
$$(4x)^2 = 64.$$

\textsc{Solution}: The first important thing is to notice that 
because the $4x$ is surrounded by brackets, which means 
we first need to square root both sides,

\begin{align*}
    \sqrt{(4x)^2} &= \sqrt{64}, \\
    4x &= \pm 8.
\end{align*}

Next to simplify $4x$ into $x$ we need to divide by 4, but
we need to remember that $\pm 8$ really represents 8 or $-8$.
Because of this when we divide by 4 we need to do it to both 
of the values on the RHS,

\begin{align*}
    \frac{4x}{4} &= \frac{-8}{4},\frac{8}{4}, \\
    x &= -2, 2. \\
\end{align*}

Then once again we can simplify $-2, 2$ into $\pm 2$
and so
$$x = \pm 2.$$

\vspace{3mm}
\hrule
\vspace{3mm}

\textsc{Example 2}: Solve the following equation,
$$(x - 4)^2 = 49.$$

\textsc{Solution}: First we need to square root both 
sides of the equation,

\begin{align*}
\sqrt{(x - 4)^2} &= \sqrt{49}, \\
x - 4 &= \pm 7.
\end{align*}

Next we will add 4 to both sides, but with addition and 
subtraction it is even more important we separate $\pm 7$ 
into $-7$ and $7$,

\begin{align*}
x - 4 + 4 &= -7 + 4, 7 + 4, \\
x &= -3, 11.
\end{align*}

Notice that because of the last step we have two different
solutions that cannot be combined using the $\pm$ symbol, so
we will just leave them separate.

\vspace{3mm}
\hrule
\vspace{3mm}

\textsc{Example 3}: Solve the following equation,
$$3x + 4 = \frac{25}{3x + 4}.$$

\textsc{Solution}: First we have $x$ on both sides of the 
equation, which we need to rectify. Since $3x+4$ is the 
denominator of the fraction on the RHS, we can multiply both
sides by $3x+4$.

\begin{align*}
(3x + 4) \times (3x + 4) &= \frac{25}{3x + 4} \times (3x + 4), \\
(3x + 4)^2 &= 25.
\end{align*}

Now that we have a squared side, we will square root both
sides

\begin{align*}
\sqrt{(3x + 4)^2} &= \sqrt{25}, \\
3x + 4 &= \pm 5.
\end{align*}

Since our next step in simplifying the RHS will be subtracting 
4, we need to separate $\pm 5$ into $-5$ and $5$ to ensure 
we account for both solutions,

\begin{align*}
3x + 4 - 4 &= -5 - 4, 5 - 4, \\
3x &= -9, 1, \\
\frac{3x}{3} &= \frac{-9}{3}, \frac{1}{3}, \\
x &= -3, \frac{1}{3}.
\end{align*}

\newpage
\subsection*{Question Bank}

\textsc{Note}: Any questions where you get a decimal as an answer
can be rounded to 2 decimal places.

\begin{enumerate}
\item Solve the following equations.
    \begin{multicols}{2}
    \begin{enumerate}
    \item .
    \end{enumerate}
    \end{multicols}


\end{enumerate}

\newpage
\subsection*{Answers}

\begin{enumerate}
\item 
    \begin{multicols}{2}
    \begin{enumerate}
    \item $x = \pm 3$
    \end{enumerate}
    \end{multicols}

\end{enumerate}

\end{document}
