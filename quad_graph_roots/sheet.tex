\documentclass[a4paper,12pt]{article}

\usepackage{worksheet}

\usepackage{tikz}

\begin{document}

\section*{Graphing Quadratics using Roots}

\subsection*{Example Problems}

\textsc{Example 1}: Draw a graph of the quadratic equation $y = (x + 2)^2$, 
and mark all important features.

\textsc{Solution}: First let's find the roots of the graph---when the graph 
crosses the $x$ axis---by making $y = 0$, 
$$0 = (x+2)^2,$$
this will be true when $x = -2$. This means our graph will cross the 
$x$ axis only once, and that for our quadratic this must be the turning 
point. Then we want to find the $y$-intercept, this will occur when 
$x = 0$,
\begin{align*}
y &= (0 + 2)^2, \\
y &= 2^2, \\
y &= 4.
\end{align*}
With this information we can plot the key points and sketch the graph,

\begin{tikzpicture}[scale=0.5]
    \draw[very thin,gray!25] (-10,-10) grid (10,10);
    \draw[thick,stealth-stealth] (-10.5,0) -- (10.5,0) node[right] {$x$};
    \draw[thick,stealth-stealth] (0,-10.5) -- (0,10.5) node[above] {$y$};
    \fill (-2,0) circle (0.2) node[below] {$(-2,0)$};
    \fill (0,4) circle (0.2) node[right] {$(0,4)$};
    \draw[very thick,domain=-5.24:1.24,stealth-stealth] plot (\x,{(\x + 2)^2});
\end{tikzpicture}

\vspace{3mm}
\hrule
\vspace{3mm}

\textsc{Example 2}: Draw a graph of the quadratic equation $y = x^2 + 2x - 8$
and label all important features.

\textsc{Solution}: First we must determine where the roots of the graph will 
be, and to do this we will factorise the quadratic expression. This gives 
$$y = (x - 2)(x + 4),$$
which tells us that the roots will occur at $x = 2$ and $x = -4$. Further 
we can find the $y$-intercept by substituting $x=0$ to get
\begin{align*}
y &= 0^2 + 2\times 0 - 8 \\
y &= -8.
\end{align*}
Finally we will find the turning point. The turning point of a parabola 
always lies on its axis of symmetry. The axis of symmetry can be calculated 
either by finding the halfway point between the roots $-4$ and 2---which would 
be $-1$---or by using the formula $\frac{-b}{2}$ which is also $-1$. We can 
then substitute the axis of symmetry as $x$ to find the $y$ coordinate of the 
turning point.
\begin{align*}
y &= (-1)^2 + 2 \times -1 - 8, \\
y &= 1 - 2 - 8, \\
y &= -9.
\end{align*}
Then we can plot these points and sketch the graph.

\begin{tikzpicture}[scale=0.5]
    \draw[very thin,gray!25] (-10,-10) grid (10,10);
    \draw[thick,stealth-stealth] (-10.5,0) -- (10.5,0) node[right] {$x$};
    \draw[thick,stealth-stealth] (0,-10.5) -- (0,10.5) node[above] {$y$};
    \fill (-4,0) circle (0.2) node[below left] {$(-4,0)$};
    \fill (2,0) circle (0.2) node[below right] {$(2,0)$};
    \fill (0,-8) circle (0.2) node[right] {$(0,-8)$};
    \fill (-1,-9) circle (0.2) node[below left] {$(-1,-9)$};
    \draw[very thick,domain=-5.42:3.42,stealth-stealth] plot (\x,{(\x - 2)*(\x + 4)});
\end{tikzpicture}

\vspace{3mm}
\hrule
\vspace{3mm}

\textsc{Example 3}: Draw a graph of the quadratic equation $y = -x^2 + x + 6$
and label all important features.

\textsc{Solution}: First we might notice that the coefficient of $x^2$ is 
$-1$, which means our parabola will be concave down---the arms will point 
downwards. To factorise the quadratic expression we will factor $-1$ out 
of the trinomial expression then factorise it like any other quadratic,
\begin{align*}
y &= -(x^2 - x - 6), \\
y &= -(x - 3)(x + 2). \\
\end{align*}
From this we can determine that the roots will be $x = -2$ and $x = 3$.
We can find the $y$-intercept by substituting $x = 0$, 
\begin{align*}
y &= -0^2 + 0 + 6, \\
y &= 6.
\end{align*}
Finally, the $x$ coordinate of the axis of symmetry will be $0.5$, so we 
can find the $y$ coordinate of the turning point to be 
\begin{align*}
y &= -0.5^2 + 0.5 + 6, \\
y &= 6.25.
\end{align*}
Then we can plot the key points and sketch the graph.

\begin{tikzpicture}[scale=0.5]
    \draw[very thin,gray!25] (-10,-10) grid (10,10);
    \draw[thick,stealth-stealth] (-10.5,0) -- (10.5,0) node[right] {$x$};
    \draw[thick,stealth-stealth] (0,-10.5) -- (0,10.5) node[above] {$y$};
    \fill (-2,0) circle (0.2) node[below left] {$(-2,0)$};
    \fill (3,0) circle (0.2) node[below right] {$(3,0)$};
    \fill (0,6) circle (0.2) node[left] {$(0,6)$};
    \fill (0.5,6.25) circle (0.2) node[above right] {$(0.5,6.25)$};
    \draw[very thick,domain=-3.59:4.59,stealth-stealth] plot (\x,{-(\x - 3)*(\x + 2)});
\end{tikzpicture}

\newpage
\subsection*{Question Bank}

% \textsc{Note}: Any questions where you get a decimal or fraction 
% as an answer can be rounded to 2 decimal places or left as 
% an exact value.

\begin{enumerate}
\item For each of the following quadratic equations find their roots, 
$y$-intercept, and turning point. Then graph the equation and label all 
the important points.
    \begin{multicols}{2}
    \begin{enumerate}
    \item $y = x^2$
    \item $y = x^2 + 3x + 2$
    \item $y = x^2 + 2x + 1$
    \item $y = x^2 - 4x + 3$
    \item $y = x^2 - 6x + 9$
    \item $y = x^2 + 11x + 28$
    \item $y = x^2 - 14x + 45$
    \item $y = x^2 - 3x - 4$
    \item $y = x^2 + 2x - 24$
    \item $y = x^2 - x - 56$
    \item $y = -x^2 - 9x - 20$
    \item $y = -x^2 - 2x - 1$
    \item $y = -x^2 + 11x - 24$
    \item $y = -x^2 + 8x - 16$
    \item $y = -x^2 + 2x + 35$
    \item $y = -x^2 - 3x + 10$
    \end{enumerate}
    \end{multicols}
\end{enumerate}

% \newpage
% \subsection*{Answers}
% 
% \begin{enumerate}
% \item 
% \end{enumerate}

\end{document}
