\documentclass[a4paper,12pt]{article}

\usepackage{worksheet}

\usepackage{tikz}
\usetikzlibrary{graphs,quotes,shapes.geometric,arrows.meta}

\newcommand{\ra}{$\to$ }

\begin{document}

\large
\section*{Shortest Path and Minimum Spanning Tree}

\subsection*{Question Bank}

\textsc{Note}: Remember that some graphs have many minimum spanning 
trees. So if your tree looks different to a solution you can check 
that the total of all edge weights on your tree is as small as 
the example tree. 

\begin{enumerate}
\item For the following graph:

\begin{tikzpicture}[every circle node/.style=draw]
    \node (A) at (2,3) [circle] {A};
    \node (B) at (6,3) [circle] {B};
    \node (C) at (0,1.5) [circle] {C};
    \node (D) at (4,1.5) [circle] {D};
    \node (E) at (8,1.5) [circle] {E};
    \node (F) at (2,0) [circle] {F};
    \node (G) at (6,0) [circle] {G};
    \graph[edge quotes={fill=white,inner sep=1mm}] {
        (A) --["2"] (B);
        (C) --["2"] (D) --["4"] (E);
        (A) --["5"] (C);
        (A) --["3"] (D);
        (B) --["5"] (E);
        (G) --["2"] (F) --["3"] (D) --["6"] (G) --["5"] (E);
    };
\end{tikzpicture}
    \begin{enumerate}
    \item Find the shortest path from C to G.
    \item Find the shortest path from F to B.
    \item Draw a minimum spanning tree.
    \end{enumerate}

\item For the following graph:

\begin{tikzpicture}[every circle node/.style=draw,scale=1.3]
    \node (A) at (1,3.5) [circle] {A};
    \node (B) at (3,3.5) [circle] {B};
    \node (C) at (5,3.5) [circle] {C};
    \node (D) at (0,1.73) [circle] {D};
    \node (E) at (2,1.73) [circle] {E};
    \node (F) at (4,1.73) [circle] {F};
    \node (G) at (6,1.73) [circle] {G};
    \node (H) at (1,0) [circle] {H};
    \node (I) at (3,0) [circle] {I};
    \node (J) at (5,0) [circle] {J};
    \graph[edge quotes={fill=white,inner sep=1mm}] {
        (A) --["3"] (B) --["4"] (C);
        (D) --["4"] (A) --["3"] (E);
        (B) --["5"] (F) --["7"] (C) --["4"] (G);
        (D) --["5"] (E) --["2"] (F) --["4"] (G);
        (D) --["3"] (H);
        (E) --["4"] (I) --["6"] (F) --["5"] (J) --["3"] (G);
        (H) --["2"] (I) -- ["4"] (J);
    };
\end{tikzpicture}
    \begin{enumerate}
    \item Find the shortest path from C to D.
    \item Find the shortest path from H to B.
    \item Draw a minimum spanning tree.
    \end{enumerate}

% remove if necessary
\newpage
\item For the following graph:

\begin{tikzpicture}[every circle node/.style=draw,scale=1.35]
    \node (A) at (0,3) [circle] {A};
    \node (B) at (2,3) [circle] {B};
    \node (C) at (4,3) [circle] {C};
    \node (D) at (6,3) [circle] {D};
    \node (E) at (0,0) [circle] {E};
    \node (F) at (2,0) [circle] {F};
    \node (G) at (4,0) [circle] {G};
    \node (H) at (6,0) [circle] {H};
    \graph[edge quotes={fill=white,inner sep=1mm}] {
        (B) --["7"] (C) --["7"] (D);
        (A) --["5"] (E);
        (A) --["5",pos=0.75] (F);
        (A) --["3"] (G);
        (B) --["4",pos=0.75] (E);
        (B) --["5",pos=0.25] (G);
        (C) --["5",pos=0.25] (F);
        (C) --["5"] (G);
        (C) --["3"] (H);
        (D) --["2"] (H);
        (E) --["9"] (F) --["8"] (G) --["6"] (H);
    };
\end{tikzpicture}
    \begin{enumerate}
    \item Find the shortest path from E to C.
    \item Find the shortest path from D to E.
    \item Draw a minimum spanning tree.
    \end{enumerate}

\item For the following graph:

\begin{tikzpicture}[every circle node/.style=draw, scale=1.1]
    \node (A) at (0,8) [circle] {A};
    \node (B) at (2,9) [circle] {B};
    \node (C) at (4,7) [circle] {C};
    \node (D) at (8,8) [circle] {D};
    \node (E) at (10,9) [circle] {E};
    \node (F) at (2.3,4.8) [circle] {F};
    \node (G) at (6,5) [circle] {G};
    \node (H) at (8.8,5.3) [circle] {H};
    \node (I) at (0,2.7) [circle] {I};
    \node (J) at (3,2) [circle] {J};
    \node (K) at (7.5,2.5) [circle] {K};
    \node (L) at (11,3) [circle] {L};
    \node (M) at (5,0) [circle] {M};
    \graph[edge quotes={fill=white,inner sep=1mm}] {
        % MST
        (A) --["8",bend left=15] (B) --["13"] (C) --["15",bend right=10] (G);
        (I) --["10"] (J) --["16",pos=0.7] (F) --["16"] (G);
        (M) --["17"] (K) --["14"] (G);
        (K) --["14"] (L) --["12"] (H) --["15"] (D) --["17"] (E);
        % Other edges
        (B) --["17",bend left=10] (E) --["18",bend right=13] (C);
        (A) --["20",pos=0.4] (C) --["25",pos=0.4] (H);
        (A) --["19"] (F) --["18",bend right=10,pos=0.4] (B);
        (D) --["20"] (G) --["19"] (H);
        (A) --["24"] (I) --["17"] (F);
        (E) --["21"] (H) --["17"] (K);
        (E) --["18",bend left=10] (L) --["21",bend left=15] (M);
        (I) --["24",bend right=15,pos=0.6] (G) --["18"] (M);
        (I) --["20",bend right=15] (M) --["19"] (J);
    };
\end{tikzpicture}
    \begin{enumerate}
    \item Find the shortest path from A to K.
    \item Find the shortest path from I to E.
    \item Draw a minimum spanning tree.
    \end{enumerate}

\item Consider that we want to try a simple greedy algorithm for finding 
the shortest path. In our new algorithm we will walk along the 
shortest edge leading out from the node we are currently at, that takes 
us to a node we have yet to visit. This algorithm will likely work for 
many graphs, but can you think of an example graph where this algorithm 
would work? and a graph where it would not?

\item The following graph is an abstract representation of the cost, due to 
toll roads, of driving from one city to another.

\begin{tikzpicture}[every ellipse node/.style=draw]
    \node (A) at (0,3) [ellipse] {Newcastle};
    \node (B) at (3,0) [ellipse] {Sydney};
    \node (C) at (6,3) [ellipse] {Canberra};
    \graph[edge quotes={fill=white,inner sep=1mm}] {
        (A) --["10"] (B) --["8"] (C) --["20"] (A);
    };
\end{tikzpicture}

For example the cost of travelling from Newcastle to Canberra 
would be \$18 via Sydney or \$20 directly. In order to reduce 
car through traffic in Sydney, the city plans to charge cars 
\$6 when they enter the city. This would mean a trip from 
Newcastle to Canberra via Sydney would now be \$24, and 
similarly a trip from Newcastle to Sydney would be \$16. 
Conversely, a trip from Sydney to Canberra should still only 
cost \$8.  Can you design a graph that represents this new change 
to the toll system? (You might need to use a directed graph.)

% \begin{tikzpicture}[every circle node/.style=draw]
%     \graph[edge quotes={fill=white,inner sep=1mm}] {
%     };
% \end{tikzpicture}

\end{enumerate}

\newpage
\subsection*{Answers}

\begin{enumerate}
\item
    \begin{enumerate}
    \item C \ra D \ra F \ra G, weight of 7
    \item F \ra D \ra A \ra B, weight of 8
    \item
    \begin{tikzpicture}[every circle node/.style=draw]
        \node (A) at (2,3) [circle] {A};
        \node (B) at (6,3) [circle] {B};
        \node (C) at (0,1.5) [circle] {C};
        \node (D) at (4,1.5) [circle] {D};
        \node (E) at (8,1.5) [circle] {E};
        \node (F) at (2,0) [circle] {F};
        \node (G) at (6,0) [circle] {G};
        \graph[edge quotes={fill=white,inner sep=1mm}] {
            (A) --["2"] (B);
            (C) --["2"] (D) --["4"] (E);
            (A) --["3"] (D);
            (G) --["2"] (F) --["3"] (D);
        };
    \end{tikzpicture}
    \end{enumerate}

\item
    \begin{enumerate}
    \item C \ra B \ra A \ra D, weight of 11
    \item H \ra D \ra A \ra B, weight of 10
    \item
    \begin{tikzpicture}[every circle node/.style=draw,scale=1.3]
        \node (A) at (1,3.5) [circle] {A};
        \node (B) at (3,3.5) [circle] {B};
        \node (C) at (5,3.5) [circle] {C};
        \node (D) at (0,1.73) [circle] {D};
        \node (E) at (2,1.73) [circle] {E};
        \node (F) at (4,1.73) [circle] {F};
        \node (G) at (6,1.73) [circle] {G};
        \node (H) at (1,0) [circle] {H};
        \node (I) at (3,0) [circle] {I};
        \node (J) at (5,0) [circle] {J};
        \graph[edge quotes={fill=white,inner sep=1mm}] {
            (A) --["3"] (B) --["4"] (C);
            (A) --["3"] (E);
            (E) --["2"] (F);
            (D) --["3"] (H);
            (E) --["4"] (I); 
            (J) --["3"] (G);
            (H) --["2"] (I) -- ["4"] (J);
        };
    \end{tikzpicture}
    \end{enumerate}

\item
    \begin{enumerate}
    \item E \ra A \ra G \ra C, weight of 13
    \item D \ra H \ra C \ra G \ra A \ra E, weight of 18
    \item
    \begin{tikzpicture}[every circle node/.style=draw,scale=1.35]
        \node (A) at (0,3) [circle] {A};
        \node (B) at (2,3) [circle] {B};
        \node (C) at (4,3) [circle] {C};
        \node (D) at (6,3) [circle] {D};
        \node (E) at (0,0) [circle] {E};
        \node (F) at (2,0) [circle] {F};
        \node (G) at (4,0) [circle] {G};
        \node (H) at (6,0) [circle] {H};
        \graph[edge quotes={fill=white,inner sep=1mm}] {
            (A) --["5"] (E);
            (A) --["3"] (G);
            (B) --["4",pos=0.75] (E);
            (C) --["5",pos=0.25] (F);
            (C) --["5"] (G);
            (C) --["3"] (H);
            (D) --["2"] (H);
        };
    \end{tikzpicture}
    \end{enumerate}

\item
    \begin{enumerate}
    \item A \ra C \ra G \ra K or A \ra F \ra G \ra K, weight of 49
    \item I \ra G \ra C \ra E, weight of 57
    \item
    \begin{tikzpicture}[every circle node/.style=draw, scale=1.1]
        \node (A) at (0,8) [circle] {A};
        \node (B) at (2,9) [circle] {B};
        \node (C) at (4,7) [circle] {C};
        \node (D) at (8,8) [circle] {D};
        \node (E) at (10,9) [circle] {E};
        \node (F) at (2.3,4.8) [circle] {F};
        \node (G) at (6,5) [circle] {G};
        \node (H) at (8.8,5.3) [circle] {H};
        \node (I) at (0,2.7) [circle] {I};
        \node (J) at (3,2) [circle] {J};
        \node (K) at (7.5,2.5) [circle] {K};
        \node (L) at (11,3) [circle] {L};
        \node (M) at (5,0) [circle] {M};
        \graph[edge quotes={fill=white,inner sep=1mm}] {
            % MST
            (A) --["8",bend left=15] (B) --["13"] (C) --["15",bend right=10] (G);
            (I) --["10"] (J) --["16",pos=0.7] (F) --["16"] (G);
            (M) --["17"] (K) --["14"] (G);
            (K) --["14"] (L) --["12"] (H) --["15"] (D) --["17"] (E);
        };
    \end{tikzpicture}
    \end{enumerate}

\item A simple graph for which this algorithm does work is the 
following graph if we want to find the shortest path from A to D.

\begin{tikzpicture}[every circle node/.style=draw]
    \node (A) at (0, 2) [circle] {A};
    \node (B) at (2, 4) [circle] {B};
    \node (C) at (2, 0) [circle] {C};
    \node (D) at (4, 2) [circle] {D};
    \graph[edge quotes={fill=white,inner sep=1mm}] {
        (A) --["2"] (B) --["2"] (D);
        (A) --["10"] (C) --["1"] (D);
    };
\end{tikzpicture}

A graph for which this algorithm would not might look like:

\begin{tikzpicture}[every circle node/.style=draw]
    \node (A) at (0, 2) [circle] {A};
    \node (B) at (2, 4) [circle] {B};
    \node (C) at (2, 0) [circle] {C};
    \node (D) at (4, 2) [circle] {D};
    \graph[edge quotes={fill=white,inner sep=1mm}] {
        (A) --["1"] (B) --["99"] (D);
        (A) --["10"] (C) --["1"] (D);
    };
\end{tikzpicture}

Because first we would travel A \ra B, but now we are locked 
into travelling along the edge from B \ra D with a weight of 99.

\item An example of using a directed graph to model this 

\begin{tikzpicture}[every ellipse node/.style=draw]
    \node (A) at (0,3) [ellipse] {Newcastle};
    \node (B) at (3,-2) [ellipse] {Sydney};
    \node (Be) at (3,0) [ellipse] {Syd.~Entry};
    \node (C) at (6,3) [ellipse] {Canberra};
    \graph[edge quotes={fill=white,inner sep=1mm}] {
        (A) --["20",arrows={Stealth[length=3mm]-Stealth[length=3mm]}] (C);
        (A) --["10",arrows={-Stealth[length=3mm]}] (Be);
        (C) --["8",arrows={-Stealth[length=3mm]}] (Be);
        (Be) --["6",arrows={-Stealth[length=3mm]}] (B);
        (B) --["10",bend left,arrows={-Stealth[length=3mm]}] (A);
        (B) --["8",bend right,arrows={-Stealth[length=3mm]}] (C);
    };
\end{tikzpicture}

% \begin{tikzpicture}[every circle node/.style=draw]
%     \graph[edge quotes={fill=white,inner sep=1mm}] {
%     };
% \end{tikzpicture}
\end{enumerate}

\end{document}
