\documentclass[a4paper,12pt]{article}

\usepackage{worksheet}

\usepackage{enumitem}

\usepackage{tikz}
\usetikzlibrary{graphs,graphdrawing,quotes}
\usegdlibrary{layered,force}

\newcommand{\ra}{$\to$ }

\begin{document}

\large
\section*{Dijsktra's Algorithm}

\begin{enumerate}
\item Create two sets of nodes, the first is the set of visited nodes 
and it will start empty, and the second is the set of unvisited nodes 
which will initially contain all the nodes.
\item Assign a distance to each node, for now the distance to the 
start node is 0 and every other node is infinite.
\item Select the node from the unvisited set with the lowest distance 
to be the current node. \label{dl_start}
\item For each edge that travels from the current node to an unvisited 
node, calculate the distance of travelling from the current node 
to the new node via this edge. If this new distance is less than 
the current distance to the new node, then replace the distance to 
the new node. 
\item Once all the edges leaving the current node to unvisited nodes 
have been considered, remove the current node from the unvisited set 
and add it to the visited set. \label{dl_end}
\item Repeat steps \ref{dl_start} to \ref{dl_end} until all nodes have 
been visited. Once this is done you can re-trace the path either by 
inspection or by recording the previous node alongside the distance to 
each node.
\end{enumerate}

\begin{tikzpicture}[every circle node/.style=draw,scale=1.5]
    \node (A) at (0,4) [circle] {A};
    \node (B) at (4,4) [circle] {B};
    \node (C) at (8,4) [circle] {C};
    \node (D) at (2,2) [circle] {D};
    \node (E) at (6,2) [circle] {E};
    \node (F) at (0,0) [circle] {F};
    \node (G) at (4,0) [circle] {G};
    \node (H) at (8,0) [circle] {H};
    \graph[edge quotes={fill=white,inner sep=1mm}] {
        % MST 
        (A) --["2"] (D) --["4"] (B) --["3"] (C);
        (B) --["3"] (G);
        (F) --["1"] (G) --["1"] (E) --["3"] (H);
        % other edges
        (A) --["7"] (B) --["4"] (E) --["7"] (C) --["8"] (H);
        (A) --["8"] (F) --["6"] (D) --["5"] (G) --["5"] (H);
    };
\end{tikzpicture}

\end{document}
