\documentclass[a4paper,12pt]{article}

\usepackage[T1]{fontenc}
\usepackage{geometry}
\geometry{
    total={150mm,237mm},
    left=30mm,
    top=30mm,
}
\usepackage{amsmath}
\usepackage{multicol}

\usepackage{csquotes}

\setlength{\parindent}{0pt}

\begin{document}
\large
\section*{Substitution Word Problems}

\subsection*{Example Problems}

\textsc{Example 1}: If the formula for calculating the number 
of paper plates ($p$) needed for a party, based on the 
number of guests ($n$), is $p = 3n$. How many paper plates 
will be needed for a party of 5 people?

\textsc{Solution}: First, when confronting a word problem 
it is usefel to create a list of which pronumeral represents
what part of the question. For example in the above question 
we have,

\begin{itemize}
\item $n$ is the number of guests,
\item $p$ is the number of paper plates required.
\end{itemize}

Then we can extract from the question the important
information. The most important part of any word problem
is identifying what the question wants you to find. In this 
question

\begin{displayquote}
How many paper plates will be needed
\end{displayquote}

tells us we are looking for the number of paper plates, 
represented by $p$ (recall the list we made at the start).
In order to find the value of $p$ using the formula,

$$p = 3n,$$

we will need to know the value of the other pronumerals in the 
formula, in this case we need to know the value of $n$. 
Thankfully, the phrase

\begin{displayquote}
a party of 5 people
\end{displayquote}

which tells us that the variable representing the number 
of guests ($n$) is 5. We could represent this as a simple equation

$$n = 5.$$

Once we have this information we can substitute it into 
the formula and find the appropriate value of $p$,

\begin{align*}
p &= 3n, \\
p &= 3\times 5, \\
p &= 15.
\end{align*}

Finally we might write a short sentence to answer the question,

\begin{displayquote}
15 paper plates will be required.
\end{displayquote}

\vspace{3mm}
\hrule
\vspace{3mm}

\textsc{Example 2}: 

\textsc{Solution}: 

\vspace{3mm}
\hrule
\vspace{3mm}

\textsc{Example 3}: 

\textsc{Solution}: 

\newpage
\subsection*{Question Bank}

\textsc{Note}: Any questions where you get a decimal as an answer
can be rounded to 2 decimal places.

\begin{enumerate}
\item .

\end{enumerate}

\newpage
\subsection*{Answers}

\begin{enumerate}
\item .
\end{enumerate}

\end{document}
