\documentclass[a4paper,12pt]{article}

\usepackage{worksheet}

\usepackage{csquotes}


\begin{document}

\large
\section*{Substitution Word Problems}

\subsection*{Example Problems}

% \textsc{Example 1}: If the formula for calculating the number 
% of paper plates ($p$) needed for a party, based on the 
% number of guests ($n$), is $p = 3n$. How many paper plates 
% will be needed for a party of 5 people?

\textsc{Example 1}: The number of paper plates ($p$) needed to cater a
party is determined by the number of guests who will attend ($n$). This
is calculated using the formula $p = 3n$. How many paper plates 
will be needed for a party of 5 people?

\textsc{Solution}: First, when confronting a word problem 
it is helpful to determine and note down which variable 
represents what aspect of the question or situation. 
For example in the above question we have,

\begin{itemize}
\item $n$ is the number of guests,
\item $p$ is the number of paper plates required.
\end{itemize}

Then we can extract from the question the important
information. The most important part of any word problem
is identifying what the question wants you to find. In this 
question

\begin{displayquote}
How many paper plates will be needed
\end{displayquote}

tells us we are looking for the number of paper plates, 
represented by $p$ (recall the list we made at the start).
In order to find the value of $p$ using the formula,

$$p = 3n,$$

we will need to substitute the value of the other pronumerals in 
the formula, in this case we need to know the value of $n$. 
Thankfully, the phrase

\begin{displayquote}
a party of 5 people
\end{displayquote}

which tells us that the variable representing the number 
of guests ($n$) is 5. We could represent this as a simple equation

$$n = 5.$$

Once we have this information we can substitute $n = 5$ into 
the formula and find the appropriate value of $p$,

\begin{align*}
p &= 3n, \\
p &= 3\times 5, \\
p &= 15.
\end{align*}

Finally we might write a short sentence to answer the question,

\begin{displayquote}
15 paper plates will be needed for the party.
\end{displayquote}

\vspace{3mm}
\hrule
\vspace{3mm}

\textsc{Example 2}: The formula for calculating the interior 
angle sum ($S$) of an $n$ sided polygon is $S = 180(n - 2)$.
What would be the interior angle sum of a heptagon (polygon with
7 sides)?

\textsc{Solution}: First we have the two variables involved
in this question
\begin{itemize}
\item $S$ represents the sum of all the angles inside the 
    polygon.
\item $n$ is the number of sides that the polygon has.
\end{itemize}
Since we know the polygon is a heptagon and has 7 sides 
this tells us that 
$$n = 7.$$
We can then substitute this into the equation and evaluate
the RHS to determine the value of $S$,
\begin{align*}
S &= 180(n - 2), \\
S &= 180(7 - 2), \\
S &= 180\times5, \\
S &= 900.
\end{align*}

\vspace{3mm}
\hrule
\vspace{3mm}

\textsc{Example 3}: 

\textsc{Solution}: 

\newpage
\subsection*{Question Bank}

\textsc{Note}: Any questions where you get a decimal or fraction 
as an answer can be rounded to 2 decimal places or left as 
an exact value.

\begin{enumerate}
\item For a restaurant with $n$ tables that each seat 4 people,
    the formula to calculate the maximum capacity ($C$) of the 
    restaurant is $C = 4n$. How many people would fit in a
    restaurant with 14 tables?

\item A pizzeria has two types of seating, it has some bar 
    seating that can seat 9 people, and $n$ tables that 
    each seat 5. The formula for calculating the maximum 
    capacity ($C$) of the pizzeria is $C = 5n + 9$, what
    would be the maximum capacity when there are 4 tables?

\item In order to calculate the perimeter ($P$) of a rectangle 
    you need to know its length ($l$) and width ($w$). The formula
    to calculate the perimeter is $P = 2l + 2w$. What would be
    the perimeter of a rectangle that is 6cm wide and 9cm long?

\item The formula to calculate the area ($A$) of a trapezium is
    $A = \frac{a + b}{2}h$. In this formula $a$ and 
    $b$ are the lengths of the two parallel sides, and $h$ is the 
    perpendicular height between them.
    \begin{enumerate}
    \item Find the area of a trapezium with parallel side lengths 
        of 4cm and 7cm, and a perpendicular height of 6cm.
    \item Using this alternate formula 
        $A = 0.5(ah + bh)$, calculate the 
        area of the trapezium from part a).
    \item For you, which formula, $A = \frac{a + b}{2}h$
        or $A = 0.5(ah + bh)$, was easier to use and why? Which 
        formula do you think is simpler or which do you prefer?
    \end{enumerate}

\end{enumerate}

\newpage
\subsection*{Answers}

\begin{enumerate}
\item 56 people % restaurant question

\item 29 people % pizzeria question

\item 30cm % rectangle perimeter question

\item % trapezium area question
    \begin{enumerate}
    \item 33cm$^2$
    \item 33cm$^2$
    \item Any reasonable answer is correct here, but common 
        observations might include the number of evaluation steps, 
        the presence of fractions or decimals, or how hard a 
        formula is to read.
    \end{enumerate}
\end{enumerate}

\end{document}
