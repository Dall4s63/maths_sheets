\documentclass[a4paper,12pt]{article}

\usepackage{worksheet}

\newcommand{\deriv}[3][]{\frac{d^{#1}#2}{d#3^{#1}}}

\begin{document}

\section*{Sheet Title}

\subsection*{Question Bank}

\begin{enumerate}
\item The core of derivatives is the idea of calculating the gradient of 
extremely small sections of a curve, and this is captured by the equation
$$f'(x) = \lim_{h \to 0}\frac{f(x + h) - f(x)}{h}.$$
As an example lets take the derivative of the function 
$$f(x) = x^3 + 5x.$$
Expanding the function applications in the limit gives us
$$f'(x) = \lim_{h \to 0}\frac{\left((x+h)^3 + 5(x + h)\right) - \left(x^3 + 5x\right)}{h},$$
the initial problem with this is that we are dividing by $h$, and this 
means we cannot substitute $h = 0$ to simplify our limit---since dividing 
by 0 is undefined. First let's expand and simplify the expression in the 
numerator,
\begin{align*}
f'(x) &= \lim_{h \to 0}\frac{x^3 + 3hx^2 + 3h^2x + h^3 + 5x + 5h - x^3 - 5x}{h}, \\
 &= \lim_{h \to 0}\frac{3hx^2 + 3h^2x + h^3 + 5h}{h}, \\
 &= \lim_{h \to 0}\frac{h\left(3x^2 + 3hx + h^2 + 5\right)}{h}.
\end{align*}
By factoring out $h$ in the numerator, and armed with the knowledge that $h$ 
never actually equals 0, we can simplify it from the top and the bottom,
$$f'(x) = \lim_{h\to 0}\left(3x^2 + 3hx + h^2 + 5\right).$$
Finally, since we are no longer dividing by $h$ we can evaluate the limit 
by substituting $h = 0$,
\begin{align*}
f'(x) &= 3x^2 + 3x\cdot 0 + 0^2 + 5, \\
&= 3x^2 + 5.
\end{align*}
For the following functions, find their derivatives with respect to $x$,
by applying the first principles formula.
    \begin{multicols}{2}
    \begin{enumerate}
    \item $f(x) = x^3$
    \item $f(x) = 4x$
    \item $f(x) = 2x^2 - 3x$
    \item $f(x) = x(2x + 1)$
    \item $f(x) = (x^2 + 2)(x - 4)$
    \item $f(x) = \frac{x + 2}{x}$
    \end{enumerate}
    \end{multicols}

\item Since finding a derivative from first principles is time consuming and 
annoying, we will learn some rules that we can apply with impunity. First 
we will cover the power rule, which often looks like
$$\mathrm{if~} y = x^n \mathrm{~then~} \deriv{y}{x} = nx^{n - 1}.$$
So for a function such as 
$$f(x) = x^4,\quad f'(x) = 4x^3,$$
and similarly,
$$f(x) = \frac{2}{x^3} = 2x^{-3}, \quad f'(x) = -6x^{-4} = -\frac{6}{x^4}.$$
For each of the following functions, find their derivatives by applying 
this power rule.
    \begin{multicols}{2}
    \begin{enumerate}
    \item $f(x) = x^5$
    \item $f(x) = 3x^2$
    \item $f(x) = -5x^4$
    \item $f(x) = -7x^{10}$
    \item $f(x) = \frac{x^3}{18}$
    \item $f(x) = \frac{2x^6}{9}$
    \item $f(x) = x^{-2}$
    \item $f(x) = -2x^{-4}$
    \item $f(x) = \frac{1}{x^3}$
    \item $f(x) = \frac{5}{4x^4}$
    \item $f(x) = 3x^2(2x^3 + x^2)$
    \item $f(x) = 4x^3\left(x^2 - 2x + \frac{3}{2}\right)$
    \item $f(x) = x^{\frac{1}{2}}$
    \item $f(x) = \frac{3}{4}x^{\frac{2}{3}}$
    \item $f(x) = x^{-\frac{1}{2}}$
    \item $f(x) = 5\sqrt{x}$
    \item $f(x) = 4\sqrt{x^3}$
    \item $f(x) = \frac{8}{3}x\sqrt{x}$
    \item $f(x) = \frac{2}{\sqrt{x}}$
    \item $f(x) = \frac{5}{\sqrt[3]{x^2}}$
    \end{enumerate}
    \end{multicols}

\newpage
\item hello
\end{enumerate}
\newpage
\subsection*{Answers}

\begin{enumerate}
\item
    \begin{multicols}{2}
    \begin{enumerate}
    \item $f'(x) = 3x^2$
    \item $f'(x) = 4$
    \item $f'(x) = 4x - 3$
    \item $f'(x) = 4x + 1$
    \item $f'(x) = 3x^2 - 8x + 2$
    \item $f'(x) = -\frac{2}{x^2}$
    \end{enumerate}
    \end{multicols}

\item
    \begin{multicols}{2}
    \begin{enumerate}
    \item $f'(x) = 5x^4$
    \item $f'(x) = 6x$
    \item $f'(x) = -20x^3$
    \item $f'(x) = -70x^9$
    \item $f'(x) = \frac{x^2}{6}$
    \item $f'(x) = \frac{4x^5}{3}$
    \item $f'(x) = -2x^{-3}$ or $\frac{-2}{x^3}$
    \item $f'(x) = 8x^{-5}$ or $\frac{8}{x^5}$
    \item $f'(x) = \frac{-3}{x^4}$
    \item $f'(x) = \frac{-5}{x^5}$
    \item $f'(x) = 30x^4 + 12x^3$ or $6x^3(5x + 2)$
    \item $f'(x) = 20x^4 - 8x^3 + 18x^2$ or $2x^2(10x^2 - 4x + 9)$
    \item $f'(x) = \frac{1}{2}x^{-\frac{1}{2}}$ or $\frac{1}{2\sqrt{x}}$
    \item $f'(x) = \frac{1}{2}x^{-\frac{1}{3}}$ or $\frac{1}{2\sqrt[3]{x}}$
    \item $f'(x) = -\frac{1}{2}x^{-\frac{3}{2}}$ or $-\frac{1}{2\sqrt{x^3}}$
    \item $f'(x) = \frac{5}{2\sqrt{x}}$
    \item $f'(x) = 6\sqrt{x}$
    \item $f'(x) = 4\sqrt{x}$
    \item $f'(x) = -\frac{1}{\sqrt{x^3}}$
    \item $f'(x) = -\frac{10}{3\sqrt[3]{x^5}}$
    \end{enumerate}
    \end{multicols}

\end{enumerate}

\end{document}
