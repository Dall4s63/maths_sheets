\documentclass[a4paper,10pt]{article}

\usepackage{worksheet}

\geometry{
    top=15mm,
    left=15mm,
    height=267mm,
    width=180mm,
}

\newcommand{\deriv}[3][ ]{\ensuremath\frac{d^{#1}#2}{d#3^{#1}}}
\newcommand{\derivof}[2][ ]{\ensuremath\frac{d^{#1}}{d#2^{#1}}}

\usepackage{parskip}

\begin{document}

\section*{Derivative Helper}

\begin{multicols*}{2}

{\large\textsc{Note on Notation}}

There are a few ways that you are likely to see derivatives notated in NSW HSC maths 
textbooks and exam papers. Just remember that these are not hard rules and you should 
use the notation that fits with a question or feels most comfortable to you while 
communicating your intentions clearly.

When a situation involves functions you are likely to see derivatives as 
$f'(x)$, $g'(x)$ and $f''(x)$, where the derivative is with respect to the input 
variable---usually $x$. This apostrophe mark may also be used with variables 
such as $y = 3x^2$ becomes $y' = 6x$. This mark may be pronounced ``prime'', 
so you might say ``y prime'' to mean the first derivative of $y$.

The other main form of notation is $\deriv{y}{x}$, which represents the derivative 
of $y$ with respect to $x$. Usually this is seen when equations use $y$ or some 
other variable such as $u = 3x$ gives $\deriv{u}{x} = 3$. Multiple derivatives are 
represented with index notation as $\deriv[2]{u}{x}$ would be the second derivative of 
$u$ with respect to $x$. A rarer use of this notation is $\derivof{x}$ which is an 
operator that roughly means ``take the derivative of [expression] with respect to $x$''.
For example, if we wanted the derivative of $2x^4$ we might write $\derivof{x}(2x^4) = 8x$.

{\large\rule{0em}{1.5em}\textsc{Constants and Coefficients}}

The derivative of a constant expression is 0. For example, if $u = 10$ then 
$\deriv{u}{x} = 0$. More generally for any constant $C$, 
$$\derivof{x}(C) = 0.$$

Constant coefficients are a bit different, consider a function, $f$, that is 
another function, $g$, multiplied by a constant, $C$. The derivative of 
this function, $f(x) = Cg(x)$, would be 
$$f'(x) = \derivof{x}(Cg'(x)) = C\derivof{x}(g(x)) = Cg'(x).$$
Notice that the constant coefficient doesn't change, I like to imagine it's 
just along for the ride. A more concrete example might be the function 
$f(x) = 4x^3$, in this case 4 is the constant coefficient of the function 
$g(x) = x^3$. So when we want to find $f'(x)$ we can apply the power 
rule to $x^3$ then multiply by 4, which might look like 
$$f'(x) = 4 \cdot \derivof{x}(x^3) = 4\cdot 3x^2 = 12x^2.$$

{\large\rule{0em}{1.5em}\textsc{Sums and Subtraction}}

When a function is made up of multiple expressions separated by 
addition or subtraction---which is just funky addition---you can 
find the derivative of the function, by finding the derivative of 
each of the smaller parts. For example, if $f(x) = g_1(x) + g_2(x) - g_3(x)$
then $f'(x) = g_1'(x) + g_2'(x) - g_3'(x)$. For a more concrete example 
consider $f(x) = x^2 + 3x + 9$, to find the derivative we will 
find the derivative of each part.
\begin{align*}
f'(x) &= \derivof{x}(x^2) + 3\cdot\derivof{x}(x) - \derivof{x}(9), \\
&= 2x + 3\cdot 1 - 0, \\
&= 2x + 3.
\end{align*}

\end{multicols*}

\newpage
\section*{Formulas}
\begin{multicols*}{2}
{\large\textsc{Power Rule}}

The power rule is used for expressions of a variable that have a constant 
exponent. For simple powers of $x$ use the rule 
$$y = x^n \quad \deriv{y}{x} = nx^{n-1}.$$
The more general formula uses the chain rule, and is
$$y = f(x)^n \quad \deriv{y}{x} = nf'(x)[f(x)]^{n-1}.$$

{\large\textsc{Product Rule}}

The product rule is used when a function is a product of two expressions 
in terms of $x$. The common formula is for when $y = uv$,
$$\deriv{y}{x} = u\deriv{v}{x} + v\deriv{u}{x}.$$

{\large\textsc{Chain Rule}}

The chain rule is used when finding the derivative of a function that 
has another function as its input---also called function composition.
You might see this written as, for $y = g(u)$ where $u = f(x)$, 
$$\deriv{y}{x} = \deriv{y}{u} \times \deriv{u}{x}.$$
This means to first do the derivative of the outside function with 
respect to its input function, then multiply by the derivative of the 
input function with respect to x. Another way of writing this would 
be, if $y = g(f(x))$ then,
$$y' = f'(x) \times g'(f(x))$$

{\large\textsc{Quotient Rule}}

This is used when one function is divided by another. Consider
$y = \frac{u}{v}$, the derivative of $y$ would be 
$$\deriv{y}{x} = \frac{v\deriv{u}{x} - u\deriv{v}{x}}{v^2}.$$
When applying this formula it is important to make sure you 
keep track of what functions are $u$ and $v$. Another option is to 
convert $y = \frac{u}{v}$ into $y = uv^{-1}$ and use the power 
and chain rules.

{\large\textsc{Exponentials and Logarithms}}

The derivative formulas for exponentials and logarithms tend to be 
surprisingly simple. Here are the simple cases, 
$$\derivof{x}(e^x) = e^x \textrm{~and~} \derivof{x}(\ln(x)) = \frac{1}{x}.$$
More generally you will apply the chain rule, for $y = e^{f(x)}$, you have 
$$\deriv{y}{x} = f'(x)e^{f(x)},$$
and for $y = \ln(f(x))$ you have,
$$\deriv{y}{x} = \frac{f'(x)}{f(x)}.$$

These forms are also useful when you have a base that is not $e$. 
The helpful rules to remember are,
$$a^x = e^{\ln(a)x} \mathrm{~and~} \log_a(x) = \frac{\ln(x)}{\ln(a)}.$$
% For example, consider $y = 3^{4x}$, we can convert it into $y = e^{4\ln(3)x}$.
% Then performing the derivative we get,
% $$\deriv{y}{x} = 4\ln(3)e^{4\ln(3)x} = 4\ln(3)3^{4x}.$$

{\large\textsc{Trigonometric Functions}}

The trig functions have slightly strange derivatives. The basic versions are,
\begin{align*}
y = \sin(x) &\mathrm{~gives~} \deriv{y}{x} = \cos(x), \\
y = \cos(x) &\mathrm{~gives~} \deriv{y}{x} = -\sin(x), \\
y = \tan(x) &\mathrm{~gives~} \deriv{y}{x} = \sec^2(x), \\
\end{align*}
Notice that the derivatives of cos and sin eventually make a loop. 
The chain rule versions of these are,
\begin{align*}
y = \sin(f(x)) &\mathrm{~gives~} \deriv{y}{x} = f'(x)\cos(f(x)), \\
y = \cos(f(x)) &\mathrm{~gives~} \deriv{y}{x} = -f'(x)\sin(f(x)), \\
y = \tan(f(x)) &\mathrm{~gives~} \deriv{y}{x} = f'(x)\sec^2(f(x)), \\
\end{align*}

{\large\textsc{Inverse Trig Functions}}

These functions have the strangest derivative formulas. The simple versions 
are 
\begin{align*}
y = \sin^{-1}(x) &\mathrm{~gives~} \deriv{y}{x} = \frac{1}{\sqrt{1-x^2}}, \\
y = \cos^{-1}(x) &\mathrm{~gives~} \deriv{y}{x} = -\frac{1}{\sqrt{1-x^2}}, \\
y = \tan^{-1}(x) &\mathrm{~gives~} \deriv{y}{x} = \frac{1}{1+x^2}, \\
\end{align*}
The chain rule versions are
\begin{align*}
y = \sin^{-1}(f(x)) &\mathrm{~gives~} \deriv{y}{x} = \frac{f'(x)}{\sqrt{1-[f(x)]^2}}, \\
y = \cos^{-1}(f(x)) &\mathrm{~gives~} \deriv{y}{x} = -\frac{f'(x)}{\sqrt{1-[f(x)]^2}}, \\
y = \tan^{-1}(f(x)) &\mathrm{~gives~} \deriv{y}{x} = \frac{f'(x)}{1+[f(x)]^2}, \\
\end{align*}

\end{multicols*}

\end{document}
