\documentclass[a4paper,10pt]{article}

\usepackage{worksheet}

\geometry{
    top=15mm,
    left=15mm,
    height=267mm,
    width=180mm,
}

\newcommand{\deriv}[3][ ]{\ensuremath\frac{d^{#1}#2}{d#3^{#1}}}
\newcommand{\derivof}[2][ ]{\ensuremath\frac{d^{#1}}{d#2^{#1}}}

\usepackage{parskip}

\begin{document}

\section*{Derivative Helper}

\begin{multicols*}{2}

{\large\textsc{Note on Notation}}

There are a few ways that you are likely to see derivatives notated in NSW HSC maths 
textbooks and exam papers. Just remember that these are not hard rules and you should 
use the notation that fits with a question or feels most comfortable to you.

When a situation involves functions you are likely to see derivatives as 
$f'(x)$, $g'(x)$ and $f''(x)$, where the derivative is with respect to the input 
variable---usually $x$. This apostrophe mark may also be used with variables 
such as $y = 3x^2$ becomes $y' = 6x$. This mark may be pronounced ``prime'', 
so you might say ``y prime'' to mean the first derivative of $y$.

The other main form of notation is $\deriv{y}{x}$, which represents the derivative 
of $y$ with respect to $x$. Usually this is seen when equations use $y$ or some 
other variable such as $u = 3x$ gives $\deriv{u}{x} = 3$. Multiple derivatives are 
represented with index notation as $\deriv[2]{u}{x}$ would be the second derivative of 
$u$ with respect to $x$. A rarer use of this notation is $\derivof{x}$ which is an 
operator that roughly means ``take the derivative of [expression] with respect to $x$''.
For example, if we wanted the derivative of $2x^4$ we might write $\derivof{x}(2x^4) = 8x$.


\end{multicols*}

\end{document}
